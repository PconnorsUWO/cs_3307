\documentclass[12pt]{article}
\usepackage[margin=1in]{geometry}
\usepackage{hyperref}      % For clickable links in the PDF
\usepackage{tocloft}       % For customizing the TOC (if desired)

\begin{document}

% Title Page
\begin{titlepage}
    \centering
    \vspace*{1in}
    {\huge \bfseries Stage \#2\\[1ex]
    Group \#57 Project Requirements Modeling: \\ UCI Chess Engine\par}
    \vspace{1.5in}
    {\Large
    Alexander King Perocho\\[0.3em]
    Vivian Guo\\[0.3em]
    Martin Hung Nguyen\\[0.3em]
    Ethan Duy Thanh Trinh\\[0.3em]
    Patrick Cauilan Connors\par}
    \vspace{1.5in}
    {\Large COMPSCI 3307 - Section 001\\[0.3em]
    Mr. Marios-Stavros Grigoriou\\[0.3em]
    28 Feb 2025\par}
    \vfill
\end{titlepage}

% Auto-updating Table of Contents
\tableofcontents
\newpage

\section{Introduction}

Dear TA,

Below is an overview of the scope for our CS3307 group project assignment, designed to lay the groundwork for our agile development process.

\subsection{Assignment Purpose \& Scope}

\textbf{Requirement Formalization:} \\
Our team has developed user stories that capture the essential, optional, and wish-listed features for our project. Each story follows a concise format—detailing the user role, desired functionality, and the resulting benefit. Additionally, every user story includes clear acceptance criteria and an estimated story point value to gauge complexity and effort.

\textbf{Initial Design:} \\
We have created a detailed UML class diagram outlining the key classes, their attributes (with types and visibility), methods (with parameters and return types), and relationships (associations, hierarchies, etc.). This diagram reflects our initial design thinking and helps ensure that our implementation will cover all major requirements.

\textbf{Project Tracking Setup:} \\
To effectively manage and track our progress, we have set up a Jira project with a Kanban board. Each user story is entered into Jira with its corresponding story points and is assigned to a team member—ensuring accountability and a clear development workflow. We have also integrated Bitbucket for change tracking.

\subsection*{Key Deadlines \& Requirements}

\textbf{Due Date:} \\
The assignment is due on Friday, February 28, 2025, by 11:55\,pm via OWL, with a late submission policy in place (20\% mark reduction per day up to two days).

\textbf{Group Collaboration:} \\
This project stage is a collaborative effort, with each team member contributing equally. While we encourage discussion with other groups for idea exchange, the final deliverable is solely our group’s work.

\textbf{Deliverables:}
\begin{itemize}
    \item \textbf{User Stories Document (PDF):} Contains all user stories with acceptance criteria and story point estimates.
    \item \textbf{UML Class Diagram (PDF):} Illustrates our proposed class design, including attributes, methods, and relationships.
    \item \textbf{Jira Board Setup:} Our Kanban board is fully populated with our user stories, with proper assignments and status updates.
    \item \textbf{Bitbucket Repository:} Used for version control and change tracking throughout the project.
\end{itemize}

\subsection{Overall Goal}

This stage aims to ensure that we have a clear, well-documented understanding of the project requirements and initial design, setting the stage for effective development and agile progress tracking. We look forward to demonstrating how our planned approach will guide our project towards successful implementation.

Thank you for your time and consideration. We are open to any feedback or questions you might have regarding our approach.

Sincerely,\\[1ex]
\textbf{Team 57}
\newpage

\section{User Stories}
Below are the user stories detailing each feature’s title, priority, user story, acceptance criteria, and estimated story points.

\subsection{1.1.1 FEN Parsing}
\textbf{Title:} FEN Parsing\\
\textbf{Priority:} High\\
\textbf{User Story:}\\
As a chess GUI user, I want to correctly parse FEN strings so that I can load and analyze my chess positions accurately.\\
\textbf{Acceptance Criteria:}
\begin{itemize}
    \item The engine correctly interprets FEN strings and sets up the board state accordingly.
    \item Valid FEN positions (e.g., castling rights, en passant targets, move counts) are handled without error.
    \item An error message is returned for invalid FEN strings.
    \item The engine parses FEN strings into bitboards correctly.
    \item The engine can generate a correct FEN string from any given board position.
\end{itemize}
\textbf{Story Points:} 3

\bigskip
\hrule
\bigskip

\subsection{1.1.2 Board Representation}
\textbf{Title:} Board Representation\\
\textbf{Priority:} High\\
\textbf{User Story:}\\
As a developer, I want to represent the chessboard using bitboards so that move generation and evaluation are efficient.\\
\textbf{Acceptance Criteria:}
\begin{itemize}
    \item The chessboard and associated attack/occupancy masks are stored using 64-bit bitboard representation.
    \item The board supports a standard 8x8 chess layout.
    \item Separate bitboards are implemented for each piece type and color.
    \item The board representation allows for efficient legal move calculations.
    \item Special moves (castling, en passant, promotions) are supported.
    \item The board state updates correctly after moves.
\end{itemize}
\textbf{Story Points:} 5

\bigskip
\hrule
\bigskip

\subsection{1.1.3 Move Generation}
\textbf{Title:} Move Generation\\
\textbf{Priority:} High\\
\textbf{User Story:}\\
As a chess engine, I want to generate all legal moves from a given position so that I can analyze and play valid chess moves.\\
\textbf{Acceptance Criteria:}
\begin{itemize}
    \item All legal moves, including special moves (castling, en passant, promotions), are correctly generated.
    \item Illegal moves (such as moving into check) are filtered out.
    \item Precomputed attack masks are cached and utilized during move generation.
    \item Magic bitboards are implemented for sliding pieces (rooks, bishops, and queens).
    \item Moves are encoded and decoded using a 20-bit scheme.
\end{itemize}
\textbf{Story Points:} 8

\bigskip
\hrule
\bigskip

\subsection{1.1.4 Search Algorithm (Alpha-Beta Pruning)}
\textbf{Title:} Search Algorithm (Alpha-Beta Pruning)\\
\textbf{Priority:} High\\
\textbf{User Story:}\\
As a chess engine, I want to search the move tree efficiently so that I can select the best possible move.\\
\textbf{Acceptance Criteria:}
\begin{itemize}
    \item A Minimax search algorithm with Alpha-Beta pruning is implemented to reduce the number of evaluated positions.
    \item The search depth is configurable by the user.
    \item The engine selects the best move based on evaluation scores.
    \item Move ordering techniques (e.g., prioritizing captures and checks) are applied to search the most promising moves first.
    \item Heuristic scores are used to propagate and determine the optimal move.
\end{itemize}
\textbf{Story Points:} 8

\bigskip
\hrule
\bigskip

\subsection{1.1.5 Position Evaluation}
\textbf{Title:} Position Evaluation\\
\textbf{Priority:} High\\
\textbf{User Story:}\\
As a chess engine, I want to evaluate a position using a heuristic function so that I can determine the relative strength of a move.\\
\textbf{Acceptance Criteria:}
\begin{itemize}
    \item The evaluation function assigns values based on material balance, piece activity, king safety, pawn structure, and center control.
    \item Evaluation outputs are consistent with established chess heuristics.
    \item The evaluation function is designed for easy tuning and optimization.
    \item It provides scored moves to the search algorithm for move propagation.
\end{itemize}
\textbf{Story Points:} 5

\bigskip
\hrule
\bigskip

\subsection{1.1.6 UCI Protocol Support}
\textbf{Title:} UCI Protocol Support\\
\textbf{Priority:} High\\
\textbf{User Story:}\\
As a user, I want the engine to support the UCI protocol so that I can use it with GUI applications like Arena or ChessBase.\\
\textbf{Acceptance Criteria:}
\begin{itemize}
    \item The engine processes standard UCI commands (uci, isready, position, go, stop, quit) correctly.
    \item Appropriate UCI responses are sent for each command.
    \item The engine successfully communicates with a UCI-compatible GUI.
\end{itemize}
\textbf{Story Points:} 3

\bigskip
\hrule
\bigskip

\newpage
\subsection{1.2.1 Adjustable Skill Levels}
\textbf{Title:} Adjustable Skill Levels\\
\textbf{Priority:} Low\\
\textbf{User Story:}\\
As a user, I want to adjust the engine’s playing strength based on an ELO rating so that I can play against opponents of different skill levels.\\
\textbf{Acceptance Criteria:}
\begin{itemize}
    \item The engine allows skill adjustment via UCI parameters (e.g., \texttt{UCI\_LimitStrength} and \texttt{UCI\_Elo}).
    \item The skill level influences move selection by reducing search depth for lower levels, introducing controlled randomization, and adjusting evaluation precision.
    \item The engine mimics various ELO ratings successfully when tested against other engines.
\end{itemize}
\textbf{Story Points:} 2

\bigskip
\hrule
\bigskip

\subsection{1.2.2 Opening Book}
\textbf{Title:} Opening Book\\
\textbf{Priority:} Low\\
\textbf{User Story:}\\
As a chess engine, I want to use an opening book so that I can play established openings efficiently.\\
\textbf{Acceptance Criteria:}
\begin{itemize}
    \item The engine can load an opening book file (formats such as .bin, .pgn, .polyglot).
    \item Moves from the opening book are prioritized during the opening phase.
    \item The engine can switch between using the opening book and calculated search based on position depth.
    \item The opening book can be disabled via a UCI option.
\end{itemize}
\textbf{Story Points:} 2

\bigskip
\hrule
\bigskip

\newpage
\subsection{1.2.3 Endgame Tables}
\textbf{Title:} Endgame Tables\\
\textbf{Priority:} Low\\
\textbf{User Story:}\\
As a chess engine, I want to use precomputed endgame tablebases so that I can play perfectly in simplified positions.\\
\textbf{Acceptance Criteria:}
\begin{itemize}
    \item The engine supports Syzygy tablebases for endgame play.
    \item It correctly queries tablebases to determine win/draw/loss outcomes.
    \item When applicable, tablebase moves are prioritized over search.
    \item Tablebase functionality can be enabled or disabled via a UCI option.
\end{itemize}
\textbf{Story Points:} 2

\bigskip
\hrule
\bigskip

\subsection{1.2.4 Multithreading}
\textbf{Title:} Multithreading\\
\textbf{Priority:} Medium\\
\textbf{User Story:}\\
As a developer, I want the engine to utilize multiple CPU cores so that move search is faster.\\
\textbf{Acceptance Criteria:}
\begin{itemize}
    \item The engine supports multithreaded search by distributing work across CPU cores.
    \item The number of threads is configurable via a UCI option (e.g., \texttt{Threads}).
    \item Performance benchmarks indicate a significant speedup compared to single-threaded execution.
    \item The implementation avoids race conditions and maintains thread safety.
\end{itemize}
\textbf{Story Points:} 3

\bigskip
\hrule
\bigskip

\newpage
\subsection{1.3.1 Machine Learning}
\textbf{Title:} Machine Learning\\
\textbf{Priority:} Low\\
\textbf{User Story:}\\
As a developer, I want to enhance the evaluation function using a machine learning model trained on grandmaster games so that the engine plays more human-like and accurately assesses positions.\\
\textbf{Acceptance Criteria:}
\begin{itemize}
    \item A neural network-based evaluation function is integrated into the engine.
    \item The model is trained on high-quality chess games (e.g., grandmaster games, high-ELO engine games).
    \item The engine can switch between handcrafted evaluation and ML-based evaluation via a UCI option.
    \item Performance tests confirm that ML-based evaluation improves the engine’s playing strength.
\end{itemize}
\textbf{Story Points:} 5

\bigskip
\hrule
\bigskip

\subsection{1.3.2 GUI Develop}
\textbf{Title:} GUI Develop\\
\textbf{Priority:} Low (Because UCI support is already provided)\\
\textbf{User Story:}\\
As a user, I want a custom graphical user interface so that I can interact with the engine without relying on third-party software.\\
\textbf{Acceptance Criteria:}
\begin{itemize}
    \item A standalone GUI is created for playing games against the engine.
    \item The GUI displays the chessboard, move list, evaluation bar, and time controls.
    \item It can send and receive moves from the engine using UCI commands.
    \item The interface is user-friendly and supports common chess features (e.g., takeback, move highlighting).
\end{itemize}
\textbf{Story Points:} 3

\bigskip
\hrule
\bigskip

\subsection{1.3.3 Online Play}
\textbf{Title:} Online Play\\
\textbf{Priority:} Low\\
\textbf{User Story:}\\
As a developer, I want the engine to connect to online chess platforms so that it can play against human opponents in real time.\\
\textbf{Acceptance Criteria:}
\begin{itemize}
    \item The engine can connect to online chess platforms (e.g., Lichess, Chess.com) using their respective APIs.
    \item It correctly handles game state updates and move submissions.
    \item The connection is stable and gracefully manages disconnections and reconnections.
    \item The implementation complies with API terms of service to ensure fair play.
\end{itemize}
\textbf{Story Points:} 5
\newpage

\section{UML Class Diagram}
% Insert or reference your UML diagram for the project.

\section{Jira Task Management}
% Detail your setup on Jira, including tasks and team assignments.

\end{document}
